\documentclass[12pt,a4paper]{article}

\usepackage[pdftex]{graphicx}
\usepackage[utf8]{inputenc}
\usepackage[hyphens]{url}
\usepackage[unicode, colorlinks=true, hypertexnames=false, allcolors=red, linkcolor=blue]{hyperref}
\usepackage[T1]{fontenc}
\usepackage{xcolor}
\usepackage{times}


\newcommand{\todo}[1]{\textcolor{red}{[[\textbf{TODO} \textbf{#1]]}}}

\begin{document}
    \begin{titlepage}
        \begin{center}
            \vspace*{1cm}
        
            \begin{figure}[h!]
                \includegraphics[scale=0.12]{VUT-FIT-logo-en.png}
            \end{figure}
            \vspace{1.5cm}

            \Large{\textbf{Epidemiological models - macro level model}} \\
            \large{Modelling and Simulation}

            \vspace{0.5cm}
                
            \vspace{1.5cm}
            
            \textbf{Abramov Mikhail (xabram00)} \\
            \textbf{Pavel Yadlouski (xyadlo00)} 

            \vfill
                
            \vspace{0.8cm}
        
            Brno University of Technologies\\
            November, 2020
                
        \end{center}
    \end{titlepage}

    \tableofcontents
    \newpage

    \section{Introduction}
    The aim of this project is to implement epidemiological model of COVID-19\footnote{https://en.wikipedia.org/wiki/Coronavirus\_disease\_2019}
    and simulate it flow using SIMLIB framework\cite{SIMLIB}. 
    Used model contains different scenarios of quarantine precautions (using different types of lockdown).
    Based on simulations of this scenarios, influence of particular scenario is shown. 
    As an experiment, theoretical scenarios form the article \todo{Should we write this?} and current lockdown type in Czech Republic are analyzed.

    \todo{We want to prove or not prove efficiency of lockdowns}
    \subsection{Contributors}    
    This project is solved by team of two students: Abramov Mikhail and Pavel 
    Yadlouski.

    \subsection{Model validation}
    Results of theoretical scenarios simulation are compared with reference results from the article. Experiment with lockdown type in Czech Republic is compared with public 
    
    From  \href{http://perchta.fit.vutbr.cz/vyuka-ims/16}{documentation requirements} \\
    \todo{kapitola 1.2: V jakém prostředí a za jakých podmínek probíhalo experimentální ověřování validity modelu – pokud čtenář/zadavatel vaší zprávy neuvěří ve validitu vašeho modelu, obvykle vaši práci odmítne už v tomto okamžiku.}
    
    \section{Sources}
    Article with the mathematical model \cite{article} for implementation is 
    found in VUT online library \footnote{https://www-sciencedirect-com.ezproxy.lib.vutbr.cz}
    
    \section{Model}
    \section{Implementation}
    \section{Experiment}
    \todo{Misha napishy suda}
    \section{Conclusion}
    \todo{Co se naucili}
    \todo{Doporuceni}

    \todo{Experiment with other country}
    \clearpage
	\bibliographystyle{abbrv}
	\bibliography{citation}
\end{document}